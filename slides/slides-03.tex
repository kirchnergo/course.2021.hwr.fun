% Created 2021-03-26 Fri 08:53
% Intended LaTeX compiler: pdflatex
\documentclass[t]{beamer}
\usepackage[utf8]{inputenc}
\usepackage[T1]{fontenc}
\usepackage{graphicx}
\usepackage{grffile}
\usepackage{longtable}
\usepackage{wrapfig}
\usepackage{rotating}
\usepackage[normalem]{ulem}
\usepackage{amsmath}
\usepackage{textcomp}
\usepackage{amssymb}
\usepackage{capt-of}
\usepackage{hyperref}
\mode<beamer>{\usetheme{Amsterdam}}
\mode<beamer>{\usecolortheme{rose}}
\usepackage{fontspec}
\usepackage{polyglossia}
\setmainlanguage[babelshorthands=true]{german}
\usepackage{hyperref}
\usepackage{color}
\usepackage{xcolor}
\usepackage[misc]{ifsym}
\definecolor{darkblue}{rgb}{0,0,.5}
\definecolor{darkgreen}{rgb}{0,.5,0}
\definecolor{islamicgreen}{rgb}{0.0, 0.56, 0.0}
\definecolor{darkred}{rgb}{0.5,0,0}
\definecolor{mintedbg}{rgb}{0.95,0.95,0.95}
\definecolor{arsenic}{rgb}{0.23, 0.27, 0.29}
\definecolor{prussianblue}{rgb}{0.0, 0.19, 0.33}
\definecolor{coolblack}{rgb}{0.0, 0.18, 0.39}
\hypersetup{colorlinks=true, breaklinks=true, anchorcolor=blue,linkcolor=white, citecolor=islamicgreen, filecolor=darkred,  urlcolor=darkblue}
\usepackage{booktabs}
\usepackage{pgf}
\usepackage{minted}
\RequirePackage{fancyvrb}
\DefineVerbatimEnvironment{verbatim}{Verbatim}{fontsize=\scriptsize}
\usetheme{default}
\author{Göran Kirchner\thanks{e\_kirchnerg@doz.hwr-berlin.de}}
\date{2021-03-26}
\title{Funktionale Programmierung in F\# (3)}
\subtitle{Grundlagen \& Funktionales Design}
\hypersetup{
 pdfauthor={Göran Kirchner},
 pdftitle={Funktionale Programmierung in F\# (3)},
 pdfkeywords={},
 pdfsubject={},
 pdfcreator={Emacs 27.1 (Org mode 9.4.4)}, 
 pdflang={English}}
\begin{document}

\maketitle

\section{Ziel }
\label{sec:org83e3ac2}

\begin{frame}[label={sec:org5463128}]{Programm}
\begin{itemize}
\item Hausaufgaben
\item Prinzipien des funktionalen Designs
\item Refactoring (Übung)
\end{itemize}
\end{frame}

\section{Hausaufgaben }
\label{sec:org03c38bb}
\begin{frame}[label={sec:orgf94adee},fragile]{Queen Attack}
 \begin{minted}[bgcolor=mintedbg,frame=none,framesep=0pt,mathescape=true,fontsize=\scriptsize,breaklines=true,linenos=false,numbersep=5pt,gobble=0]{fsharp}
open System
let create (row, col) = row >= 0 && row < 8 && col >= 0 && col < 8
let canAttack (queen1: int * int) (queen2: int * int) = 
    let (r1, c1) = queen1
    let (r2, c2) = queen2
    Math.Abs(r1 - r2) = Math.Abs(c1 - c2) || r1 = r2 || c1 = c2
let whiteQueen1, blackQueen1 = (2, 2), (1, 1)
let test1 = canAttack blackQueen1 whiteQueen1
let whiteQueen2, blackQueen2 = (2, 4), (6, 6)
let test2 = canAttack blackQueen2 whiteQueen2
[test1; test2]
\end{minted}

\begin{verbatim}
val create : row:int * col:int -> bool
val canAttack : int * int -> int * int -> bool
val whiteQueen1 : int * int = (2, 2)
val blackQueen1 : int * int = (1, 1)
val test1 : bool = true
val whiteQueen2 : int * int = (2, 4)
val blackQueen2 : int * int = (6, 6)
val test2 : bool = false
\end{verbatim}
\end{frame}

\begin{frame}[label={sec:org7a72229},fragile]{Raindrops}
 \begin{minted}[bgcolor=mintedbg,frame=none,framesep=0pt,mathescape=true,fontsize=\scriptsize,breaklines=true,linenos=false,numbersep=5pt,gobble=0]{fsharp}
let rules =
    [ 3, "Pling"
      5, "Plang"
      7, "Plong" ]
let convert (number: int): string =
    let divBy n d = n % d = 0
    rules
    |> List.filter (fst >> divBy number)
    |> List.map snd
    |> String.concat ""
    |> function
       | "" -> string number
       | s -> s
let test = convert 105
test
\end{minted}

\begin{verbatim}
val rules : (int * string) list = [(3, "Pling"); (5, "Plang"); (7, "Plong")]
val convert : number:int -> string
val test : string = "PlingPlangPlong"
\end{verbatim}
\end{frame}

\begin{frame}[label={sec:orgd9a2859},fragile]{Gigasecond}
 \begin{minted}[bgcolor=mintedbg,frame=none,framesep=0pt,mathescape=true,fontsize=\scriptsize,breaklines=true,linenos=false,numbersep=5pt,gobble=0]{fsharp}
let add (beginDate : System.DateTime) = beginDate.AddSeconds 1e9
let test = add (DateTime(2015, 1, 24, 22, 0, 0)) = (DateTime(2046, 10, 2, 23, 46, 40))
test
\end{minted}

\begin{verbatim}
val add : beginDate:DateTime -> DateTime
val test : bool = true
\end{verbatim}
\end{frame}

\begin{frame}[label={sec:org4693a46},fragile]{Bank Account (1)}
 \begin{minted}[bgcolor=mintedbg,frame=none,framesep=0pt,mathescape=true,fontsize=\scriptsize,breaklines=true,linenos=false,numbersep=5pt,gobble=0]{fsharp}
type OpenAccount =
    { mutable Balance: decimal }
type Account =
    | Closed
    | Open of OpenAccount
let mkBankAccount() = Closed
let openAccount account =
    match account with
    | Closed -> Open { Balance = 0.0m }
    | Open _ -> failwith "Account is already open"
\end{minted}

\begin{verbatim}
type OpenAccount =
  { mutable Balance: decimal }
type Account =
  | Closed
  | Open of OpenAccount
val mkBankAccount : unit -> Account
val openAccount : account:Account -> Account
\end{verbatim}
\end{frame}

\begin{frame}[label={sec:org40062e0},fragile]{Bank Account (2)}
 \begin{minted}[bgcolor=mintedbg,frame=none,framesep=0pt,mathescape=true,fontsize=\scriptsize,breaklines=true,linenos=false,numbersep=5pt,gobble=0]{fsharp}
let closeAccount account =
    match account with
    | Open _ -> Closed
    | Closed -> failwith "Account is already closed"
let getBalance account =
    match account with
    | Open openAccount -> Some openAccount.Balance
    | Closed -> None
let updateBalance change account =
    match account with
    | Open openAccount ->
        lock (openAccount) (fun _ ->
            openAccount.Balance <- openAccount.Balance + change
            Open openAccount)
    | Closed -> failwith "Account is closed"
\end{minted}

\begin{verbatim}
val closeAccount : account:Account -> Account
val getBalance : account:Account -> decimal option
val updateBalance : change:decimal -> account:Account -> Account
\end{verbatim}
\end{frame}

\begin{frame}[label={sec:orgfba2a2d},fragile]{Bank Account (3)}
 \begin{minted}[bgcolor=mintedbg,frame=none,framesep=0pt,mathescape=true,fontsize=\scriptsize,breaklines=true,linenos=false,numbersep=5pt,gobble=0]{fsharp}
let account = mkBankAccount() |> openAccount
let updateAccountAsync =        
    async { account |> updateBalance 1.0m |> ignore }
let ``updated from multiple threads`` =
    updateAccountAsync
    |> List.replicate 1000
    |> Async.Parallel 
    |> Async.RunSynchronously
    |> ignore
let test1 = getBalance account = (Some 1000.0m)
test1
\end{minted}

\begin{verbatim}
val account : Account = Open { Balance = 1000.0M }
val updateAccountAsync : Async<unit>
val ( updated from multiple threads ) : unit = ()
val test1 : bool = true
\end{verbatim}
\end{frame}

\begin{frame}[label={sec:org2346c6b}]{Pause}
\begin{block}{}
You’re bound to be unhappy if you optimize everything.

\null\hfill -- Donald Knuth
\end{block}
\end{frame}

\section{Prinzipien des Funktionalen Designs }
\label{sec:org7afb2c6}
\begin{frame}[label={sec:orgcc0319b}]{Funktionales Design}
\(\leadsto\) \href{./3.1 Functional Design Patterns.pdf}{Functional Design Patterns}
\end{frame}

\begin{frame}[label={sec:org089c18f}]{Prinzipien (1)}
\begin{itemize}
\item Funktionen sind Daten!
\item überall Verkettung (Composition)
\item überall Funktionen
\item Typen sind keine Klassen
\item Typen kann man ebenfalls verknüpfen (algebraische Datentypen)
\item Typsignaturen lügen nicht!
\item statische Typen zur Modellierung der Domäne (später mehr;)
\end{itemize}
\end{frame}

\begin{frame}[label={sec:org48a8169}]{Prinzipien (2)}
\begin{itemize}
\item Parametrisiere alles!
\item Typsignaturen sind "Interfaces"
\item Partielle Anwendung ist "Dependency Injection"
\item Monaden entsprechen dem "Chaining of Continuations"
\begin{itemize}
\item bind für Optionen
\item bind für Fehler
\item bind für Tasks
\end{itemize}
\item "map" - Funktionen
\begin{itemize}
\item Nutze "map" - Funktion von generische Typen!
\item wenn man einen generischen Typ definiert, dann auch eine "map" - Funktion
\end{itemize}
\end{itemize}
\end{frame}

\begin{frame}[label={sec:org6c5a8a0}]{Pause}
\begin{block}{}
If we’d asked the customers what they wanted, they would have said “faster horses”.

\null\hfill -- Henry Ford
\end{block}
\end{frame}

\section{Refactoring }
\label{sec:org52cdd10}
\begin{frame}[label={sec:org42bb9c4},fragile]{Tree Building (Übung)}
 \begin{minted}[bgcolor=mintedbg,frame=none,framesep=0pt,mathescape=true,fontsize=\scriptsize,breaklines=true,linenos=false,numbersep=5pt,gobble=0]{shell}
exercism download --exercise=tree-building --track=fsharp
\end{minted}
\end{frame}

\begin{frame}[label={sec:org2cdbc3a},fragile]{Tree Building (Imperativ)}
 \begin{minted}[bgcolor=mintedbg,frame=none,framesep=0pt,mathescape=true,fontsize=\scriptsize,breaklines=true,linenos=false,numbersep=5pt,gobble=0]{fsharp}
let buildTree records =
    let records' = List.sortBy (fun x -> x.RecordId) records
    if List.isEmpty records' then failwith "Empty input"
    else
        let root = records'.[0]
        if (root.ParentId = 0 |> not) then
            failwith "Root node is invalid"
        else
            if (root.RecordId = 0 |> not) then failwith "Root node is invalid"
            else
                let mutable prev = -1
                let mutable leafs = []
                for r in records' do
                    if (r.RecordId <> 0 && (r.ParentId > r.RecordId || r.ParentId = r.RecordId)) then
                        failwith "Nodes with invalid parents"
                    else
                        if r.RecordId <> prev + 1 then
                            failwith "Non-continuous list"
                        else
                            prev <- r.RecordId
                            if (r.RecordId = 0) then
                                leafs <- leafs @ [(-1, r.RecordId)]
                            else
                                leafs <- leafs @ [(r.ParentId, r.RecordId)]
\end{minted}
\end{frame}

\begin{frame}[label={sec:orgc958f70},fragile]{Tree Building (Funktional)}
 \begin{minted}[bgcolor=mintedbg,frame=none,framesep=0pt,mathescape=true,fontsize=\scriptsize,breaklines=true,linenos=false,numbersep=5pt,gobble=0]{fsharp}
let buildTree records = 
    records
    |> List.sortBy (fun r -> r.RecordId)
    |> validate
    |> List.tail
    |> List.groupBy (fun r -> r.ParentId)
    |> Map.ofList
    |> makeTree 0

let rec makeTree id map =
    match map |> Map.tryFind id with
    | None -> Leaf id
    | Some list -> Branch (id, 
        list |> List.map (fun r -> makeTree r.RecordId map))
\end{minted}
\end{frame}

\begin{frame}[label={sec:org365a5cd},fragile]{Tree Building (Error Handling)}
 \begin{minted}[bgcolor=mintedbg,frame=none,framesep=0pt,mathescape=true,fontsize=\scriptsize,breaklines=true,linenos=false,numbersep=5pt,gobble=0]{fsharp}
let validate records =
    match records with
    | [] -> failwith "Input must be non-empty"
    | x :: _ when x.RecordId <> 0 -> 
        failwith "Root must have id 0"
    | x :: _ when x.ParentId <> 0 -> 
        failwith "Root node must have parent id 0"
    | _ :: xs when xs |> List.exists (fun r -> r.RecordId < r.ParentId) -> 
        failwith "ParentId should be less than RecordId"
    | _ :: xs when xs |> List.exists (fun r -> r.RecordId = r.ParentId) -> 
        failwith "ParentId cannot be the RecordId except for the root node."
    | rs when (rs |> List.map (fun r -> r.RecordId) |> List.max) > (List.length rs - 1) -> 
        failwith "Ids must be continuous"
    | _ -> records
\end{minted}
\end{frame}

\begin{frame}[label={sec:org7a4fa1a},fragile]{Tree Building (Benchmarking)}
 \begin{itemize}
\item \href{https://github.com/dotnet/BenchmarkDotNet}{BenchmarkDotNet}
\end{itemize}

\begin{minted}[bgcolor=mintedbg,frame=none,framesep=0pt,mathescape=true,fontsize=\scriptsize,breaklines=true,linenos=false,numbersep=5pt,gobble=0]{shell}
dotnet run -c release
\end{minted}

\begin{minted}[bgcolor=mintedbg,frame=none,framesep=0pt,mathescape=true,fontsize=\scriptsize,breaklines=true,linenos=false,numbersep=5pt,gobble=0]{shell}
sed -n 508,520p $benchmarks
\end{minted}

\tiny
// * Summary *

BenchmarkDotNet=v0.12.1, OS=macOS 11.2.2 (20D80) [Darwin 20.3.0]
Intel Core i7-7920HQ CPU 3.10GHz (Kaby Lake), 1 CPU, 8 logical and 4 physical cores
.NET Core SDK=5.0.101
  [Host]     : .NET Core 5.0.1 (CoreCLR 5.0.120.57516, CoreFX 5.0.120.57516), X64 RyuJIT DEBUG
  DefaultJob : .NET Core 5.0.1 (CoreCLR 5.0.120.57516, CoreFX 5.0.120.57516), X64 RyuJIT


\begin{center}
\begin{tabular}{lllllrrrlll}
Method & Mean & Error & StdDev & Median & Ratio & RatioSD & Gen 0 & Gen 1 & Gen 2 & Allocated\\
\hline
Baseline & 10.113 μs & 0.1998 μs & 0.3849 μs & 10.074 μs & 1.00 & 0.00 & 3.6163 & - & - & 14.8 KB\\
Mine & 6.539 μs & 0.2335 μs & 0.6812 μs & 6.354 μs & 0.63 & 0.06 & 2.0828 & - & - & 8.52 KB\\
\end{tabular}
\end{center}
\end{frame}

\section{Ende }
\label{sec:org7f195f8}
\begin{frame}[label={sec:org4148e83}]{Zusammenfassung}
\begin{itemize}
\item funktionales Design
\item funktionales Refactoring
\end{itemize}
\end{frame}

\begin{frame}[label={sec:org8ff54b1}]{Links}
\begin{itemize}
\item \href{https://fsharp.org/}{fsharp.org}
\item \href{https://docs.microsoft.com/de-de/dotnet/fsharp/}{docs.microsoft.com/../dotnet/fsharp}
\item \href{https://sergeytihon.com/}{F\# weekly}
\item \href{https://fsharpforfunandprofit.com/}{fsharpforfunandprofit.com}
\item \href{https://github.com/fsprojects/awesome-fsharp}{github.com/../awesome-fsharp}
\end{itemize}
\end{frame}

\begin{frame}[label={sec:org3cc3253}]{Hausaufgabe}
\begin{itemize}
\item exercism.io (E-Mail bis 16.4)
\begin{itemize}
\item[{$\square$}] Accumulate
\item[{$\square$}] Space Age
\item[{$\square$}] Poker (Programmieraufgabe)
\end{itemize}
\end{itemize}
\end{frame}
\end{document}